\documentclass{article}
\usepackage{epigraph}
\title{6.033 Lecture 4 \\ Operating Systems Part IV}
\author{Americo De Filippo}
\begin{document} 
  \maketitle
  We are going to get enfored modularity via client-server organization.
  The last time we have seen how the linker works with modularity, 
  he perform some operation with more programs and outputs some result
  that is the exec file. It implemented name mapping with Table mapping, 
  Search throught contexts.
  \section{Interface between modules}
    The different modules interact with one another via \textbf{procedure contract}.
    \paragraph{When you call a procedure} The caller and callee interact with each other 
      via returning arguments in the registers. The important thing is that this contract
      has to obej to a principle called \textbf{stack discipline}, the callee has to leave
      the stack has it has founded when been called.
    \paragraph{how can the discipline be violeted} The Callee corrupts the stacks, Callee
      crashes. This is called fate sharing, so this modularity is \textbf{soft}, cause there
      is not shielding between called and callee.
  \section{Client and Server organization}
    Taking as example the previus situation, the Client is the caller and the Server is 
    the callee. The basic abstraction is that the two are running on different machine so
    they are in some way indipendent, for this reason for make them communicate we have to 
    implement some kind of way of communication between the two. We are going to have a 
    Server-Client way of communicating (take TCP as example), in this way the two are 
    indipendent, for this reason the fail of one do not interfere with the stack of the 
    other, this we call \textbf{enforced modularity}. One problem with this implementation
    is the case which we want to decide what something exacly happends, for example we 
    are not going to know whether a client is taking long time processing something or
    is crashed. This is one of the trade-offs that we have to make for have a more 
    secure system.
  \section{Enforced modularity}
    For solving the problem of knowing what is happening, enforeced modularity comes
    with a timer called \textbf{Watchdog timer}, the server is going to have an idea of what is 
    happening at the client, if the server does not return in a precise amount of time 
    the server is going to ask himself whether the client has failed or is computing.
    Another nice proprety of this organization which is that clients geeet enforced modularity,
    the idea is that many client can access the same server but they are indipendent from each
    other. So you can you the server to has a \textbf{trusted intermidiary}, the clients 
    do not trust each other but all of them trust the server.
  \section{How to implement this stuff}
   There are differnt ways of implmenting, all of them are different mainly by ways that they
   send messages for making the communication. A pretty standard way of implementing is called
   \textbf{remote procedure call (RPC)}. An RPC implementation that is used is: \textbf{XML RPC}
   \paragraph{how to we solve the problem of dead vs slow?} We have three ways of communicating 
     that we are gonna call Exacly-once semantic (one request-one return), At-least once semantic
     (meaning that we are gonna try until we get a success). The latter is good when the 
     server has idem-port semantic (when you run an operation more than once you get the same
     answer). The third is called At-Most once (if you dont get a responce you will handle 
     the fail directly, if you get a responce good!) 
     \paragraph{async procedure call} for performance feature we are gonna have an async 
      communication (like in TCP), let's say that the client makes some requests it will not 
      be there and wait instead it will go on with other task and it will handle the responce
      along with a request number (sequence number in TCP) that enables the two to be even more
      decouple.
    \paragraph{Intermediary} The client could send a request to the server when he its down, in
      this case there will be some kind of intermidiary that will handle the requests at behalf
      of the server that then is going to handle and compute the request.
\end{document}
