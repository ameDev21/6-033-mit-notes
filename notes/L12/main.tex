\documentclass{article}
\usepackage{epigraph}
\title{6.033 Lecture 12 \\ Networking IV}
\author{Americo De Filippo}
\begin{document} 
  \maketitle
  \begin{thebibliography}{9}
    \bibitem{texbook}
    Saltzer, Jerome H. and M. Frans Kaashoek. Principles of Computer System Design: An Introduction (2009): \textbf{Section: 7.8}
  \end{thebibliography}
  Today we are going to end the networking section, and the last argument is the 
  congestion control. The last time we talked about reliability, timers, sliding window and
  flow control. All this concept are contained into the E2E layer.
  \section{Sharing and Congestion Control}
    Today we are gonna see how to achieve sharing of link and of course sharing the resources. 
    So imagine you have a bunch of computer that share the network for communicating with
    each other (you don't want a specific link for every computer). Imagine that the links
    have 1000 Mbits/sec of bandwidth. Instead the pipe which is at the between of the net
    have a bandwindth of 1 Mbits/sec (which we are gonna call the bottomneck link).
    \subsection{Cross-layer}
      \underline{Plan}: Sender i sends message at rate $r_i$ [bits/sec] \\ If they are sending
      Too fast there must be some way to make the sender to slow down, otherwise if he is 
      sending too slow there must be a way of telling him to send faster.
      \paragraph{Buffering} At any point in time you could have multiple packet arrving to
        a point, but he can handle just one of them. So in order to not drop the packet we
        use buffers. Now the question is: \textbf{How much?} How much should my buffer be 
        sized? We cannot have too small size cause it will end up dropping a lot of packets, 
        on the other hand he cannot be too big cause he will slow down the link. That could 
        cause \underline{congestion collapse}
        \subsubsection{Congestion Collapse}
          You could end up with a really big queue. The E2E sender is trying to decise if the
          packet has been dropped or is still in the queue. But when the time will go up 
          he will resend, in this case you will have multiple copy of the packet in the same
          queue. Causing congestion collapse.
          \subsection{The Law $\sum_{i=1}^{N} L_i < C$}
            If this is achieved we are gonna have: \\ 1. No congestion collapse \\ 2. Reasonable
            utiliz'n (efficiency) \\ 3. Equitable allocation.
            \paragraph{Small infraction of this Law} We will handle the infraction of this law
              if the RTTs are less then 100 RTTs with buffering, without bothering telling the
              sender to slow down. Otherwise if it will be more than 100RTTs we will send a 
              feedback to the sender and it will change the sending speed. One way of sending
              feedback to the sender is by dropping packets aways (sign of congestion miss of 
              ACK for that packet), therefore it will make the Cong. Window /= 2.
    \subsection{How to start a communication? Slow Start}
      The congestion window is handle in the following way: On each ACK you will increase 
      the window size for $2^n$ with n is the number of ACK which is an exponential increase.
      After a while it will automatically decise (based on some fraction of the number where
      last congestion happened) to increase the window linearly, then perhaps some congestion 
      happen and the we will half the congestion window size.
\end{document}
