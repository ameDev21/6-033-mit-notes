\documentclass{article}
\usepackage{epigraph}
\title{6.033 Lecture 23 \\ Security IV}
\author{Americo De Filippo}
\begin{document} 
  \maketitle
    \begin{thebibliography}{9}
      \bibitem{texbook}
      Saltzer, Jerome H. and M. Frans Kaashoek. Principles of Computer System Design: An Introduction (2009): \textbf{Section: 11.4 - .6} 
    \end{thebibliography}
    \maketitle
    In this lecture we are gonna wrap up our discussion about security, and indeed this will be 
    the last lecture of this course. The last time we talked about building a secure comm channel
    throught authorization, that will be handle in 2 ways: or ACLs or Tickets.
    \section{Athentication Protocol}
      From the last example where B send some request to W trought a secure communication channel.
      The thing that we are gonna focus on is: how is gonna know B that he has the right public key
      or W, and no one is impersonating W. We (as usual) are gonna use a certificate authority, 
      that is gonna tell to B is that is the right key of W. Now is the question is: How B is gonna
      trust the cetificate authority, how is gonna enstablish the communication with it?
      \subsection{Authentication Logic (BAN logic)}
        What BAN is gonna do is basically give us reasons to trust some certificate authority.
        One thing to do this is by a "web of trust", let's say that we trust B then we are gonna
        ask B is he knows some people that he trustes that can verify that a private key of A is 
        actually what is saying to be.
        \paragraph{Make Assumptions Explicit}  
          \begin{itemize}
            \item Assume signature is not forgeable
            \item Assuming private keys are actually private
          \end{itemize}
          This assumptions are usen when for example, we call sign(m, $k_A$), I infer from 
          this that A says m, also I'm assuming that the private key of A is actually private.
        \paragraph{Establishing initial trust}
          \begin{itemize}
            \item The way of doing this is by the previus metioned method (web of trust).  
            \item Does W know/trust P? \\ How does it decide? Usually for some data that is good.
              Like for amazon.com could be the credit card company gives some kind of allowence 
              to trust the user. 
            \item How does P trust W? Usually on the internet is via a certificate authority, from 
              this arise issues. Like to know how do i trust the certificate authority? On the 
              internet usually or you allow to the system a specific list of an exernal certificate
              authority. Or in most of the case an operating system comes out with a default list of 
              certificate that the machine trusts. Also when you install a browser it will come out
              with a list of certificate authority that it trust.
          \end{itemize}
\end{document}
