\documentclass{article}
\usepackage{epigraph}
\title{6.033 Lecture 6 \\ Operating Systems Part VI}
\author{Americo De Filippo}
\begin{document} 
  \maketitle
    \begin{thebibliography}{9}
    \bibitem{texbook}
    Saltzer, Jerome H. and M. Frans Kaashoek. Principles of Computer System Design: An Introduction (2009): \textbf{Sections: 5.1 / .3}
  \end{thebibliography}
  \section{Virtual Processors VP}
    The idea is to create a processor that runs the programs on his own
    process, indipendent of the others programs that are running. In order to 
    have this, we are going to introduce the notion of a \textbf{thread}, which is
    a collection of instruction that the modules in running as well as the current 
    state of the program PC and the stack pointer and the heap which is shared between
    threads.
    \subsection{How is gonna implement this?} The threads will run one at time, in this
      way we are going to switch between the threads while saving the progress of a threads
      in a slice of time.
    \subsection{Cooperative scheduling} In this approch each thread will tell at the processor
      'im done executing' and the processor will switch to another module, the obvius problem 
      here is that the thread will never stops and the processor will block on a single module.
    \subsection{Preemptive scheduling} The kernel in this case decide when stopping a thread 
      and run another.
    \subsection{Yield() (routine for coopering scheduling)} Yield() is what the current thread
      calls when give up the processor, so yield() is going to save the state and schedule the 
      next thread to run and dispach the next thread.
    \subsection{Thread scheduler} He is gonna decide which thread run next, is gonna have 
      an array of all the threads in the system, an integer for the next thread index and
      id of the current executing thread. Look at the slide for an example of thread stack
      handling.
    \subsection{Pipes}
      The pipes allows the programs to communicate using the procedures from the file system
      call interface. This will communicate using the following two procedures SEND() RECEIVE().
      Usually in UNIX we want to have this thing writing in a file, in this case we will
      use a bounded buffer so that many threads (also on different physical cores) can
      access to the buffer and send some messages or receive it.
    \subsection{Changing to Preemptive Scheduling} We have no explicit Yield() statements, 
      the idea is that we have a special timer that clocks periodicly into the kernel, this
      timer is gonna be connected to what is called an \textbf{interrupt} (causes the processor to run
      a special piece of code). This timer is be gonna check by the microprocessor and if is
      true the microprocessor is gonna run a special instruction. The way that is gonna work 
      is that when the kernel switch thread with the Yield(on the current thread) forcing it
      to stop.
  \section{Coopering in different address spaces} When we talk about process we mean an address
    space (virtual) + some collection of threads. The kernel provides routines to create and
    process and to destroy a process. The create() in gonna allocate the address space into 
    the table (page-map) and is going to add the code into the address space, is gonna create
    a thread for the new process add to the thread table, setup PC etc... \footnote{look at the
    slides}
    \subsection{Hierarcy of threads} In this way we can make an entire tree of threads that 
      doesnt have to know of the others threads in the same layer, but the ones that are blow 
      the layer have to know which threads is running and have to handle the threads operations.
    \subsection{Cooperating Access between threads} let's say we are going to have global
      data structure that is shared between two threads, we want to handle the access on
      this global variable? So the thing that we want to do is to check for specifics 
      charateristic on the data structure that the two threads have to do.
    \subsection{Sequence coordination operators} Allow threads that are waiting for a condition
      to run. In order to introduce this we are going to add a new data type: eventcount, integer
      that indicates the number of time that something has occurred and we are going to 
      introduce two new routine; Wait(eventcount, value) if (eventcount <= value), Notify(eventcount)
      notify wake up everybody who is waiting on the variable passed.
\end{document}
