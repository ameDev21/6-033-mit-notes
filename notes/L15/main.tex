\documentclass{article}
\usepackage{epigraph}
\title{6.033 Lecture 15 \\ Fault Tolerance II}
\author{Americo De Filippo}
\begin{document} 
  \maketitle
  \begin{thebibliography}{9}
    \bibitem{texbook}
    Saltzer, Jerome H. and M. Frans Kaashoek. Principles of Computer System Design: An Introduction (2009): \textbf{Section: 9.1}
  \end{thebibliography}
  \maketitle
  One of the most important concept in this area of computer science is \textbf{Atomiciy}.
  \paragraph{Failures} One way of dealing with it is via Replicate a component and then vote.
    But the problem with it is that often is really expencive and do not work always.
  \paragraph{Recoverability} Allow to fail, but in order that you can detect the failure 
    and you start some kind of recover procedure. Let's say a child want to learn to walk 
    one way could have been, make the child never fall, another way instead could (and is)
    make the child fall but giving him a procedure to stand up again.
  \section{The transfer problem}
    xfer(from, to, \$ amount), the approch is "all or nothing" (no middle states). So during 
    the recover procedure the system has to have a way of backingout in order to abort the
    action made before the failure.
    \subsection{Concurrency}
      Image that you have 2 transfer running at the same time, on the same data item. As we 
      now we have to be careful with the order of execution of the 2 threads. So you would like
      to see some kind of ordered actions, this is done via \textbf{Isolation} of actions 
      \footnote{"Do all before or all after"}.
      Of course we don't want to run one after the other (we'd lose concurrency).
  \section{Atomcity}
    The basic idea is to hide the fact that some sequence of action are composed. Another
    interpretation is that atomic means: REC + ISO.
    \subsection{Recoverability}
      General plan is to design module to be fail-fast. Once the failure is detected you run
      some kind of repair procedure and it restarts (allow the module to continue working).
      \paragraph{The Golden Rule} Never modify the only copy.
      \begin{enumerate}
        \item Recoverable sector: coming out with a scheme that will allow us to do read/write
          on a sector, that will allow us to resuscitate our system.
        \item Version history: this will recoverable sector as bootstrap, and are going to
          build up on for being able to have a general solution.
        \item Logs
      \end{enumerate}
    \subsection{Model}
      The first assumption that we make is that there is not concurrency, we'll also assume
      that there is no hardware errors (check sum for every sector) and there are only 
      software errors.
      \subsection{careful\_put, \_get} 
        This two procedure will enable us to have a system that is able to do not worry about 
        hardware failure, this is achieved by controlling the sectors in which is allowed 
        to read and to write. We won't allow anyone to access sectors which are ambiguos or
        half written.
      \subsection{Commit point}
        Before the commint point every one that want to read the data will get the older version
        of it, instead after the version is updated.
\end{document}
