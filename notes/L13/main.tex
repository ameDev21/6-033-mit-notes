\documentclass{article}
\usepackage{epigraph}
\title{6.033 Lecture 13 \\ Distibuted Naming \& DNS}
\author{Americo De Filippo}
\begin{document} 
  \maketitle
  \begin{thebibliography}{9}
    \bibitem{texbook}
    Saltzer, Jerome H. and M. Frans Kaashoek. Principles of Computer System Design: An Introduction (2009): \textbf{Section: 4.4}
  \end{thebibliography}
  The DNS (domain name system) is a bridge that covers the topic of networking (that we 
  talked about) and tolerant reliable computing that we are gonna talk about.
  \section{The DNS}
    At the network layer the IP address (that are needed for naming the endpoint of a 
    connection) are name that tells you where in the net topology the computer is located. 
    But cause human are not good at remembering long number that are at the first hand 
    meaningless we instead of that have crated some kind of parallel translation that is human
    like. \\ The DNS solve this problem, what he basically does is: a map from host names to
    an IP address.
    \subsection{The Designing goals}
      The main goals of DNS are Scalability (implemeted via the Distibuited database 
      model) and Reliability.
      \subsubsection{The Distribuited Database Model}
        Which ais a federated model, where organization are managing everything about that
        namespace, and all the stuff about naming. It's a pretty nice model cause everybody 
        does some about of resouces, and this is one of the reason cause the model scaled 
        so good.
      \subsubsection{Reliability}
        Fondamentaly it is implemented via an abstraction called Lookups, that work via 
        giving a name a returing a Record (Ip address) dns\_resolve(dnsname). Application
        specify a context.
    \subsection{Namespace}
      The DNS is a structured namespace, that is organized as a tree. The root is at the top
      of domains that then are going to split into subdomains. The first layer is constitued 
      by the generic top lavel domains (com, edu, netm org ...). Every layer of the tree
      is translated into a dot, and is read from bottom to top.
    \subsection{Delagation}
      This is the reason how the system scales. The best way to understeading this is 
      recursevely. Is divided into Root (the one who ones the network), the Top-Level-Domain
      (which are the owner of .edu, .com ...), this structure is usefull cause we have only
      to go to the owner of the up label in the three and ask for creating a new down label. 
    \subsection{Records}
      This is what outputs the dns when requesting for some name. What he contains is:
      A : Address, MX : (mail exanger for email), CNAME (canincal name) : "Synonsym", NS :
      Name server record.
      \subsubsection{Name Server Record}
        let's that there is a NS record for mit.edu, that NS record gives the name of 
        the machine that is resposable for mapping the name on that zone. In this way 
        things up in the tree do not know anything about who is associated with some name
        but they know someone which now more about it which is the NS record down them. And 
        this is done recursevely.
        \paragraph{Stab Resolver} You have an application that what to call dns\_resolve(), 
        there the stab resolver send the dns request to some Name Server that help for 
        resolving some name. Usually the request is send out to the root name server, him 
        reading the name for the last character is able to handle the request by for 
        example forwarding them to .edu that them is going to forward it to .mit and so on so
        fourth. The delegation could be done also without being called with the .edu or .com
        but is could also be done using the CNAME(s) which rappresents essetialy the one
        that are willing to handle the delegation.
    \subsection{Bootstrap}
      The bootstrap is needed after having the root ns. In fact this is just being published. 
      The second issue is how dost the stub connect to any name server? The most common
      approch is just with DHCP that we'll tell which name server to use.
    \subsection{Performance}
      The approch to solve the too many round trip problem is caching. The kind of name resolution
      that is going on in DNS, the iterative resolution is done by pointers that refers you to
      the right place. Instead recursevely resolution, which is done by giving you directly 
      the answer, but having directly the answer you could actually cache it. On the other
      hand you can cache referral that are actually more useful cause you could cache the 
      address of the general zone, instead of caching the address of just some kind of specific
      machine in that zone.
      \subsubsection{Expiration Times (ttl)} It says that here is the answer and is valid for
        such and such time, so tou have to uncache this entry after the ttl ends.
\end{document}
