\documentclass{article}
\usepackage{epigraph}
\title{6.033 Lecture 5 \\ Operating Systems Part IV}
\author{Americo De Filippo}
\begin{document} 
  \maketitle
  The last time has been discussed the Client/Server model, one of the 
  most important preconcept was that teh Client and Server were on different
  machines, in this lecture we will put the model on the same machine. We are 
  going to transfer the model on a \textbf{vitual computer}, a virtual computer
  has a \textbf{virtual memory} and a \textbf{virtual processor}. The goal is to
  make it look like a client/server model.
  \section{Virtual Memory VM}
    \paragraph{Why do we want a virtualized memory?} A virtual memory is needed for isolate 
      the memory of each module, let's say that some module A can write on the memory of some 
      module B we have lost the isolation of the modules. Another problem is the debugging 
      issues that we are going to face when the memory is not addressed at each module.
    \paragraph{Solutions} You want to verify the memory access for each module that do not
      enables A to write in others module's virtual memory (are valid only in a context of 
      a given module). So for each module we are going to create an \textbf{address space 32bit} 
      which is virtual. With this organization the module cannot access to the memory location
      of some other module cause is not allowed to do so.
    \subsection{Simplified VM Hardware} \textbf{Look at the slides!} \\ When an instruction try to 
      access some virtual address, the microprocess is gonna send this to the VM system
      that is gonna resolve this in some physical address. The way that the mapping is done
      is via something called \textbf{page-map}, which is a table from virtual address 
      to physical address. So every module we have a own page-map, the job of the processor
      is to access only the page-map owner to some specific module. In this way we are going to 
      have a 'separate memory for each module'
    \subsection{How does the page-map work?} The simpler way of looking at it is like a pointer
      that tells you where the vitual memory starts for a module A. The problem with this is
      that we have to have sequential memory for each module. The idea is to take the 32 bit 
      virtual address and we are going to split in two pieces: page number and an offset. So
      now we are going to store a page of memory at each entry in the table that maps 
      for a physical address, in this way we can have not sequential memory, but instead 
      sparse memory. The offset is used for orients ourself in the page.
    \subsection{How this data structure are putted togheter?} We need something that has to 
      manage all the infrastacture, and that is called the Kernel.
  \section{Kernel}
    At this moment we needed someone that manage this data structure, that is able to create 
    new page maps and look at the module. He must be able to add new module, expands memory 
    etc... This supervisory module is called the \textbf{Kernel}. The microprocessor at this 
    point is composed by the PMAR (pointer to modules running) and we are going 
    to add the user/kernel bit (this will tells us which is executing). 
    The module is going to have special permissions in particular on rules will be that 
    only the Kernel will be able to change the PMAR. The kernel is going to be another module
    that is going to run onto the virtual computer but he is going to have all the pagemaps
    of the others modules and is going to have a table from module to page-map. Futhermore the
    Kernel can create new map, allocate new memory in the vitual addressed of some module.
    \subsection{How do we communicate with the Kernel?} How for example we request at the Kernel
      the allocation of some more virtual addresses. We are going to add the notion of a 
      \textbf{Supervisor Call} (special instruction that invoces the Kernel), he accept 
      a parameter name of a gate function (well-defined entry point for invoce a piece of code
      into another module). So when a user program execute an instruction supervisor call the
      special instruction is going to set the user/kernal bit to kernal, set the PAMR to the
      kernal page-map, save the PC to be the address of the gate function. 
    \subsection{The trusted intermidiet}
      The kernal is the one that will sit between the client for make them communicate.
      Suppose we have two processes A and B that want to communicate with each other, we
      are going to create a special set of supervisor-call for the messages exchange. In this
      case the Kernel has to verify the correctness of the messages of the two modules.
    \subsection{VM for shared libraries between modules} This practice is maked really simple 
      with the VM organization, cause the kernel is going to handle this request via
      allocationg in both the page-maps the address of the physical space of the library.
      In this case we are going to have shared memory between the module (only the kernel
      can do that).
\end{document}
