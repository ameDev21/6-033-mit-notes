\documentclass{article}
\usepackage{epigraph}
\title{6.033 Lecture 22 \\ Security III}
\author{Americo De Filippo}
\begin{document} 
  \maketitle
    \begin{thebibliography}{9}
      \bibitem{texbook}
      Saltzer, Jerome H. and M. Frans Kaashoek. Principles of Computer System Design: An Introduction (2009): \textbf{Section: 11.5} 
    \end{thebibliography}
    \maketitle
    Today we are going to talk of authorization and confidentiality. 
    The cyptographic primitives that we are gonna use are: Sing, Verifiy. \\
    sign(m, k) = sig. \\ 
    verify(m, sig, $k_2$) = outputs if m corrispondes to that signature \\
    We also talked about Encrypt and Decrypt. \\
    enc(m, $k_1$) = c \\ 
    dec(c, $k_2$) = m
    \section{Secure Communication Channel}
      \begin{itemize}
        \item use pub key to exchange a shared key 
        \item use shared key to enc. comm
      \end{itemize}
      \section{Confidentiality}
      Is the protection of information exchenge between Alice and Bob, the idea is that 
      Alice create some message encypt it, and send it over ethe internet, and arrives to Bob
      that will decrypt it with its key. The proprieties is that the people into the internet
      cannot read the message.
      \subsection{+ Authentication}
        The way that we do this is via sign a message, encypt it, and add some kind of key.
      \section{Authentication}
        Let's say that we have some browser B communicate to a Web server W through a secure 
        communication channel (SSL: secure socket layer), the channel has been authenticate from
        a CA (certificate authority). How does W know that B is authorized to access W? The issue
        is that once the protocol is enstablished they can communicate with each other, so B as
        to have some kind of protocol for knowing which information is B authorized to access.
        This will be done in 3 steps.
        \begin{enumerate}
          \item Rendezvous (setup, logging in)
          \item Verification (mediate, allowing to log)
          \item Revoke
        \end{enumerate}
        There are 2 windely used approches: Lists and Tickets \\
        \begin{tabular}{ | l | l | l |}
          \hline
          Steps & Lists & Tickets \\ \hline
          Set up & add to list & generate ticket \\ \hline
          Mediate & search list, check credentials & table lookup \\ \hline
          Revoke & remove from list & invalidate ticket \\ \hline
        \end{tabular}
\end{document}
