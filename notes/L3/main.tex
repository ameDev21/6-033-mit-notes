\documentclass{article}
\usepackage{epigraph}
\title{6.033 Lecture 3 \\ Operating Systems Part III}
\author{Americo De Filippo}
\begin{document}
  \maketitle
  \section{How the gcc compiler works with modularity?}
  The first two are also called the Linking program (LD).
  So when you run gcc for produce an actual file that will 
  run it will run in the back others programs that will do
  the follwing thing. The Linker takes as argument two kinds 
  of object file: 
  \textbf{Relocatable obj file} and a \textbf{Shared obj file} and produces
  as output and \textbf{executable file}
  \subsection{Symbol resolution and Relocation, Static Linking}
    This steps find for unresolved symbols, meaning that are not 
    specified in the modules. Each file $f_n$ contains a local sysbol table, 
    that contains a set of items $D_n$ that correspnd to local variable or functions
    that are defined in the module $f_n$. Then you have a set of thing $U_n$ that are 
    not defined in $f_n$. 
    \paragraph{Implmenting linkers} (as in linux) we are going to scan from left to right 
      building three sets. \\ The first set is called O (all the object file that go into the
      output O U $f_i$) \\ the second step is D (defined symbol) D U $D_i$, and the last is 
      the U (undefined symbol so far) U u $U_i$ - (the defined symbols in the current file
      that we are anlyzing). The linking success if the U set is null at the end. A problem
      of this implementation is a duplicate definition module. So it has to be careful
      finding some duplicate defintion methods or variable.
    \paragraph{Symbol resolution with libraries}
      A library is a bunch of obj file together, essentialy is just a concatenation
      of a lot of .o files. The Approc is essentialy the same with just one different.
      The main different is the size of the libraries are much bigger than our own 
      programs. So the process of scanning all the library would be really expensive, 
      for this reason we are going to include the .o file where are defined the 
      remains items inside the U set.
    \paragraph{The problem now is how to find the undefined symbols? }
      We have a set of names and a set of values (where is defined). This is resolved 
      through a name-mapping algorithm, that takes a context as input (seen in before lectures)
      There are Three ways of doing this: \\ The first way is a table lookup, (inode table or 
      the table of defined symbol that very .o file takes with him) \\ The Second way is a 
      path-name resolution. The third way (seen today) searching throught modules (search 
      in all the file which is done in a form of table lookup). 
  \subsection{Relocation}
    Relocation is pretty streightforward, in static linking cause it simply maintain 
    a relocation table in each file. It stores a table of pointer for \textbf{position
    dependent code}, this will enable the resolution for the addresses of the data in the
    programs.
  \subsection{Loading / Program Loading}
    The third varies a lot depending on the system. The problem solved by program
    loading is looking a what file has been requested, taking the context, putting 
    in the memory and passing the control. In UNIX is done by 'execve', it pass the 
    control to an interpreter that will invoce the first line in main (in C).
  \section{Shared object, Dynamic Linking}
    This idea solves the problem of including a library just one time when is called 
    multiple times. Why have multicopies of the same module?
    \paragraph{Problems with them} The basic problem is that instruction are allocated with
      memory allocation, the problem is that when you have two programs that want to access
      at the same module at the same time, is very hard to share the object (is position 
      dependent by where is called) 
    \paragraph{Solved by: Position Indipendent Code} When you call the module and he 
      allocates memory he will have allocation dependent to the Program Counter
      and not to the code has been called from. 
    When you have the object indipendent code, you do not have to include the .o file 
    where let's printf() is defined. Instead the object file has to maintain the file
    name from where printf() is defined (a pointer to the file). Thats called dynamic 
    linking (instead of including the complete module that contains the defintion of 
    a function we are going to maintain a reference to the position of the defintion 
    and we invoce it when needed).
\end{document}
