\documentclass{article}
\usepackage{epigraph}
\title{6.033 Lecture 10 \\ Networking II}
\author{Americo De Filippo}
\begin{document} 
  \maketitle
  \begin{thebibliography}{9}
    \bibitem{texbook}
    Saltzer, Jerome H. and M. Frans Kaashoek. Principles of Computer System Design: An Introduction (2009): \textbf{Section: 7.4}
  \end{thebibliography}
  Last time we talked about the three layers: ETE, network, link. We went about how this
  interact with each other in a connection. Today we are gonna finish the discussion of 
  the link layer, and then we are gonna go about the net layer.
  \section{The others Issue of link layer: Framing}
    When you are sending a frame (packet in the link layer) on the link layer the recevier
    has to know whether the packet is starting or ending. The solution is pretty simple 
    we are gonna to add simply some number at the start and another at the end of the 
    message. 
    \paragraph{what if the net-layer includes in his message the end code?} One solution
      could be is to make the end symbol reserved only to the link layer. Another idea is 
      called bit-stuffing, suppose we send our end code let's say 1111, the sender is gonna
      convert 111 to 1110 and the recv will do the inverse. In this way we are gonna
      ensure that when is encounter the 1111 signal we are sure that is the end symbol. 
  \section{Network layer}
    The network layer is based on the Ip protocol that has and Ipv4 Header \footnote{you could
    review this online}. An important section of the header is the Time-to-live in order to 
    waste time into loops in our network.
    \subsection{Forwarding}
      The idea is that given a packet what the next hop ought to be. They are gonna do that
      by keeping a table, that will map the destination with the next link that he should use.
      The routing is the one that is in charge of deciding with hop should choose in order to
      have the shorter path. 
    \subsection{Routing}
      As we have said before here we are gonna choose the best path in order to have
      the faster route for our packet to the destination. \\ The principles of the routing should be:
      \begin{enumerate}
        \item Scalable: We dont want people involved in configuring this millions 
          of routers.
        \item Robust: If a node fails the network should handle the fail without 
          interrupting its execution.
        \item Distributed: We want that every machine does something, we dont want a machine
          that does all the job.
      \end{enumerate}
    \paragraph{Path Vector routing} The couting will be devided in 2 steps, Advertisement and 
      the integration step. Of course when the advertisment that comes to a node has in it 
      the node id, it will ignore the adv.
    \paragraph{Error handling in path vector} Suppose that a link fail, for the 2 principle 
      it wont stop, on the other hand it will just keep going expiring the node after he does
      not hear adv from the failed node.
    \paragraph{"Soft state"} You should relie on an information only when it is regulary
      refreshed.
  \subsection{The scalability is not achevied with path-vector}
    The solution is: Hierarchy, throught sub-networks \footnote{read the recitation paper about
    it}. For example if the MIT sub-network would go down it wont affect all the others 
    sub-networks the internet. The idea is to have autonomous systems inside our enormous
    network. To the edge of two regions there will be an edge-node that will interconnect them.
\end{document}
