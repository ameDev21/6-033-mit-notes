\documentclass{article}
\usepackage{epigraph}
\title{6.033 Lecture 16 \\ Fault Tolerance III}
\author{Americo De Filippo}
\begin{document} 
  \maketitle
  \begin{thebibliography}{9}
    \bibitem{texbook}
    Saltzer, Jerome H. and M. Frans Kaashoek. Principles of Computer System Design: An Introduction (2009): \textbf{Section: 9.2 / 9.3}
  \end{thebibliography}
  \maketitle
  In this lecture we are focus on how to implement Recoverabilty (motivation for the topic in
  the last lecture).
  \paragraph{How COMMIT() is implemented and used?}
  \section{Version histories}
    The idea is the same of last time: Never modify the only copy. It generalise this idea
    in: never modify anything, create another version of it. You are gonna have a Cell storage
    and a Journal storage. When you assign a value to a cell storage you will overwrite the
    previus value. For this reason we have a Journal storage that is gonna keep track of the
    value and id of the one which has made the modification. The jornal will contain only
    the committed action, so in case of failure you will go onto the last commited action.
  \section{Logging}
    The way to think of a log is like a version history, but is data structure that will 
    contain only the read and write operation sequentially. So we are gonna use Cell storage
    for read/write, then the log is going to be stored on a non-volatile medium and is 
    sequentially.
  \subsection{The plan for recover using logging}
    When you fail the system run a recovery procedure from the log: for action that are 
    uncommitted we have to look at the updating of the cell store (the log will help us).
    For committed actions we are going to install them (redo). When ABORT() is called
    we are going to undo current action.
  \subsection{How to use the log}
    Let's say we are going to update a variable, the log is going to take some kind of 
    diary of all the update done to the system (the variable in it). A log is like an
    Append-only data structure: he is going to have two kind of record: 
    \begin{itemize}
      \item \textbf{UPDATE}: \\ id: 172 \\ undo: x = old \footnote{in case of changing the value
        the old value would become the undo, for backing down} \\ redo: x = new \footnote{this is
        where the new value is stored, in case has to be (re)done}
      \item \textbf{OUTCOME}: \\ id: 172 \\ status: COMMITED or ABORTED or PENDING
    \end{itemize}
  \subsection{When to write to log?}
    \paragraph{Disk-bound DB} Is a disk where application are writing to a database where cell
      storage are implemented. Cell storage are actually into the disk. Futhermore, you have
      to maintain a log on a different disk. So when you write something to disk you have to
      update the log record. Now the question is when writing into the log? Let's say you
      write into the variable, and then you crash before writing into the log, you are in
      truble. 
    \subsubsection{2 principle: WAL protocol and logging COMMIT()}
      The first principle is: Write ahead logging. Append to the log before writing to the 
      cell storage. Suppose now you set x to some value and then you crash, you are garateed
      that the log has been written. The second principle is to log the COMMIT record 
      before returining from the COMMIT().
    \subsubsection{How to implement ABORT()?}
      If the system calls ABORT(), it will look into the steps into the action cells and 
      go into the undone field of those, and assign it (going back to the previus version of the
      every involved cells storages).
  \subsection{Recovery procedure}
    \begin{itemize}
      \item\textbf{Scan log backwards}
      \item\textbf{Winners} COMMITED or ABORTED \\ \textbf{Losers} everything else
      \item\textbf{Redo} COMM. Winners \\ \textbf{Undo}
    \end{itemize}
\end{document}
