\documentclass{article}
\usepackage{epigraph}
\title{6.033 Lecture 21 \\ Security II}
\author{Americo De Filippo}
\begin{document} 
  \maketitle
    \begin{thebibliography}{9}
      \bibitem{texbook}
      Saltzer, Jerome H. and M. Frans Kaashoek. Principles of Computer System Design: An Introduction (2009): \textbf{Section: 11.2 - .3} 
    \end{thebibliography}
    \maketitle
    In the module of last time we had a client that was going to communicate through a network
    with a server. We have said that we had 3 problems that we were addressing:
    \begin{itemize}
      \item\textbf{Authentication}
      \item\textbf{Authorization}
      \item\textbf{Confidentiality}
    \end{itemize}
    Last time we have speak about cyptography, what it does is via some kind of encyption
    we are going to trasmit the messages that may seems meaningless and only with the key you 
    can reverse the process and have the real message. We talked about private and public key 
    protocols \footnote{perhaps read a paper on this is.}.
    \section{Authentication}
      When we autheniticate a user we want to achieve:
      \begin{enumerate}
        \item Who is requesting?
        \item Message sent == message recv 
      \end{enumerate}
      \subsection{mechanical or techincal}
        How we are going to authenticate a user? We can do this via One time pad, this is an
        XOR tequinique in order to encypt the bit stream, the problem with this is that lucifer
        can manipolate the message in some random way and perhaps change the message. So what
        we want is a checksum dependent on a key operation.
      \subsection{Key distribution problem}
        If Alice cannot phisicaly meet with Bob, she could send a message to Bob telling him 
        her private key, but at this point we have the same problem of knowing that was in 
        fact Alice to send the message. Instead we are going to use certificates, the ideas
        is that we are going to use a third in the communication that is both known to both 
        of them (Charles). Alice at this point is going to send to Bob the public key the same
        as before, at this point Bob is going to ask Charles if that is the correct public key 
      of Alice.
    \section{Secure Communication Channel or Confidentiality}
      First we are going to authenticate the user with the previus tequinique, then we are 
      going to use public key to authenticate and use a shared key protocol. There are 3
      proprieties that we want for our protocol.
      \begin{itemize}
        \item\textbf{Freshness}: Recent history message, timestamps.
        \item\textbf{Appropriate}: I'm actually the intendent recv of this message.
        \item\textbf{Forward secrecy}: We should be able to change the key and the protocl
          would still work.
      \end{itemize}
\end{document}
