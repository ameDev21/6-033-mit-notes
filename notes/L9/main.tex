\documentclass{article}
\usepackage{epigraph}
\title{6.033 Lecture 9 \\ Networking II}
\author{Americo De Filippo}
\begin{document} 
  \maketitle
  \begin{thebibliography}{9}
    \bibitem{texbook}
    Saltzer, Jerome H. and M. Frans Kaashoek. Principles of Computer System Design: An Introduction (2009): \textbf{Section: 7.2}
  \end{thebibliography}
  Review of last time about best effort, which means that there are some proprety that 
  won't be garateed.
  \section{Layering}
    As we have seen previusly a standard way of dealing with complexity is with modularity, 
    and the modularity used into network system are layers and protocols (which is just 
    a set of rules for make two device communicate in order to deal with complexity).
    We are gonna see speficiation of this layer in the next lecture starting with the end-to-end
    layer that will be in charge of the resolution of our best effort's problems.
  \section{Encapsulation}
    We are gonna achieve the layering via encapsulation layer by layer, inserting or 
    removing his header/trailer for his information. Another important charateristic is the
    independecy of each layer, every layer don't know anything about the others layers and in 
    way we are gonna achieve completely indipendecy by the modules (strong modularity).
  \section{Link Layer Issues}
    The link layer needs to provide: manager bits transmission via Digital -> Analog -> Digital
    conversion, it does framing which is basically how the software on the layer decide 
    whether a new frame has began or ended. Another of them is Channel Access which is how
    someone that want to send a message used the wire or the layer without intefering with 
    another one which is transmetting \footnote{read paper on ethernet}. The last
    thing that is done in this layer is error detection/correction, the idea is that 
    when transmitting the message on a wire the message could be corrupted or there could be 
    some problem with the transmission. 
    \subsection{D to A to D coversion}
      Suppose we have a sender that want to send some kind of binary bits flow, that could be 
      rappresented by a quadratic curve, we decide when sending a new bits every clock signal.
      But when the reciver reciev the data it will have not a perfect smooth signal and for
      this reason we want to decode the original message. This will be done knowing the frequency
      of the message \footnote{here we know that there are a lot arguments like shannon's etc...}
      But what could happen is our clock signal could be with the same frequency but they are 
      out of phase.
      \paragraph{How to fix the out of phase problem} The solution is called PLL (phase lock loop)
        so we basically need to know how much we need to shift our signal. A simple way of implementing
        this is called \textbf{oversampling 8x}, with this tequinique what we should see if we are
        perfetctly alligned in time the same number(8) of 0 or 1 for due original bits of the signal. 
        And when this does not happen we are gonna to shift to the side of the one with less number
        of continuosily 1 or 0 \footnote{see the slides (slides/L9.pdf) for more clarity}.
      \paragraph{How to fix long sequence of the same bit} If I have a long sequence of the same
        bit is that i cannot know how much should i shift my signal. The solution to this is called
        Manchester Encoding: the idea is that we are trasmitting a zero for a transition to a low
        to a high and a 1 from a high to low. In this way i wont have a flat signal. 
    \subsection{Clock Sync and Manchester Encoding}
      The issue that this enconding resolves is to have a sync clock, in order to read the 
      meassage correcly. A clock is a message that alternates from 0 to 1. With the clock
      signal we can know how many bits we have. This could not happen if we don't sync the
      two clocks. The place where we miss some bit caused by the mismatch of the two clock
      signal is called \textbf{Clock SI:P}. One way that we can sync the two clocks is via 
      gps (antenna), with atomic clocks (which are really precise).
  \end{document}
