\documentclass{article}
\usepackage{epigraph}
\title{6.033 Lecture 8 \\ Networking I}
\author{Americo De Filippo}
\begin{document} 
  \maketitle
  \begin{thebibliography}{9}
    \bibitem{texbook}
    Saltzer, Jerome H. and M. Frans Kaashoek. Principles of Computer System Design: An Introduction (2009): \textbf{Section: 7.1}
\end{thebibliography}
  A computer network is some kind of a connection that make some 
  computers commiunicates between each others. They allow us among 
  a lot of thing to access remote data, and separate client/server. 
  In the end they are a complex and interesting computer system.
  \section{Universal communication}
    The goal is the universal communication between differents machine, 
    the network system that we are gonna to describe is entailed into the
    cloud that is between all the computers. 
    \subsection{Interesing issues}
      d(tech)/dt (physical medium, percetuge message get lost), there are also
      some kind of limits that are just physical like the speed of light. Another
      issue is that they are a shared infrastructure, they are multi user usage.
    \subsection{tecnology}
      They are heterogeneous $10^7$ difference, this means that there are a lot of 
      differents protocols, or different performance systems like routers, smartphones 
      network modules for computers etc...  \\ In general we want to define some kind of 
      formula which is like $length/spdl + bits/secb$ The first term is called the bit rate
      and the second term is the propagation delay.
    \subsection{sharing}
      \paragraph{Multiplexing or switching} This is a way of sharing nodes in order to access some kind 
        other computer and is good for avoid to have all connected to everyone, which would
        cause a awful amout of wires. The idea behind this is to have swiches a various 
        location around the internet. In this way we are gonna have some kind of different 
        destination but same data stuff. For this reason we are gonna introduce the notion
        of \textbf{routing}.
        \subsubsection{Routing} 
          How to get to everybody location onto the network? This is decision made by the 
          switch based on their own table (we are gonna see this futher in the future).
        \subsection{circut switching}
          The idea for this is that we want to setup some reserved channel between 2 end 
          points. Of course we dont use this tecnology anymore, but the idea is in some way
          preserved. Let's say we have some operators A, B, C that want to communicate on the
          same wire. What our new tecnology does is something called TDM (time division 
          multiplexing), exlusively divied traffic for some kind of evenly slices time.
          Futhermore we have to know the number of intervals so that we can have some kind of
          idea of the max number of data per second. If we want to know the capacity of the
          circuit, we want to look at the max number of data that can traffic in one interval.
          \paragraph{The problem with it} The real issue is the continue nature of the 
            circuit switching, we do not know the rates of traffing on our network.
        \subsection{packet switching}
          Is simply not possible to know rate of traffic throght the network. The idea is
          that every node can talk at any moment. The complexity is that we cannot predict
          the data amout, but we have to know it cause we cannot exeed the network capacity.
          For this reason we are gonna have some kind of queue, whe the queue overfill we have
          some kind of \textbf{congestion}. If a queue overfill almost ever we have to reject
          data or drop some.
          \paragraph{Best effor networking} They dont garatee that the data will be delivered, 
            but the system will the best that he can for having the data delivered. What we 
            are gonna do in the next session is to look of how we are gonna create software  
            in order to have the best effor principle work as hard as he can.
\end{document}
