\documentclass{article}
\usepackage{epigraph}
\title{6.033 Lecture 17 \\ Fault Tolerance IV}
\author{Americo De Filippo}
\begin{document} 
  \maketitle
  \begin{thebibliography}{9}
    \bibitem{texbook}
    Saltzer, Jerome H. and M. Frans Kaashoek. Principles of Computer System Design: An Introduction (2009): \textbf{Section: 9.4 / 9.5}
  \end{thebibliography}
  \maketitle
  The topic that we are gonna tackle today is isolation. 
  The idea behind isolation is to run multiple actions at the same time, and have a result 
  which you would have if runned one after the other.
  \section{Starting with slow and correct}
    Let's say you have 2 actions $T_1$, $T_2$. The first read x and write y, the second on the
    other hand write x and write y. So we can see this steps runned in a lot of number of 
    ways. So if you look at the atomic actions, there are some that conflict with each other
    and others that don't. 
    \subsection{Search for the conflicts}
      Some usually confictual actions are r(z), w(z) and combination of them. So we are gonna 
      search for the conflictual actions and we are going to serialize. What we mean is to run
      the actions in a different orders that will make the result look just the same as the
      action would've runned sequentially.
    \subsection{Serializability}
      Trace's confict arrow should be in the same order as some serial order of actions.
      \subsubsection{Actions graphs}
        Given one traces we are gonna produce a graph and from there see if some proprety 
        holds. The nodes are the actions, the edges are the trace between the actions. If 
        two action share a conflict operation you are going to draw an edge between 
        the actions. When you draw the graph you can find a path for going all through 
        the path with the shorter path (in case of multiple serial orders). And that will 
        be the equivalent order of actions on the one in which they run sequentially.
        \paragraph{Result 1: If the graph has not cycle the order is serializable and viceversa}
          Notice: If it would happen that for run some operations we would have some conflicts
          that could be resolved. To prove this remember that if you have no cycle you
          can run a topological sort order. A topological sort is described as follows: 
          \begin{verbatim}
            L ← Empty list that will contain the sorted elements
            S ← Set of all nodes with no incoming edge

            while S is not empty do
                  remove a node n from S
                  add n to L
                  for each node m with an edge e from n to m do
                      remove edge e from the graph
                      if m has no other incoming edges then
                          insert m into S

              if graph has edges then
                  return error   (graph has at least one cycle)
              else 
                  return L   (a topologically sorted order)
            \end{verbatim}
       \subsubsection{Locks}
         Recalling if you have a variable you can do 2 actions onto this, acq(lock of x), rel("").
         With lock you can really slow down the system, you would probabily remove the concurrency
         and have a slower system. Futhermore, you could get in trouble creating cycle in the
         graph.
      \subsubsection{Simple locking}
        The idea is that every action before doing every reads or writes would acquire the locks,
        in this way no other conflicting action can be in the same state. So this protocol will
        provide isolation but it would be slow, another problem would be that every action 
        has to know all the data that it needs (there could be things like if statements that 
        can make the situation complicated).
      \subsubsection{Two phase locking}
        The idea is that there should be no release before all the acquire are done. It does
        provide isolation. How we are gonna prove this is by constructing the graph after 
        using 2PL, if that graph has no cycle the approch will be correct. 
\end{document}.
